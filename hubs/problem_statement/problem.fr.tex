\problemname{\problemyamlname}

\illustration{0.3}{code-processor.jpg}{
    CC BY-SA 4.0 by usmanabce on \href{https://www.vecteezy.com/vector-art/15807909-code-processor-vector-icon-design}{Vecteezy}
}

Le KARWa a tellement de succès cette année que, l'année prochaine, il sera nécessaire d'organiser $n$ hubs pour
accueillir les $m$ participants. Les hubs sont situés aux coordonnées $(a_i , b_i )$, avec $i = 1, ..., n$ et les participants aux coordonnées $(x_j , y_j )$, avec $j = 1, ..., m$.
Les organisateurs sont des fans du K-Means et afin de les aider à quantifier le placement des hubs, pouvez-vous
déterminer la somme de toutes les distances euclidiennes au carré entre les hubs et les participants ?
\begin{Input}
    L'entrée consiste en :
    \begin{itemize}
      \item une ligne avec un entier $n$ ($1 \leq n \leq 10^{6}$) représentant le nombre de hubs,
      \item $n$ lignes, chacune contenant deux entiers $a$ et $b$ ($0 \leq a, b\leq 10^{9}$) représentant les coordonnées du hub,
      \item une ligne avec un entier $m$ ($1 \leq m \leq 10^{6}$)  représentant le nombre de participants,
      \item $m$ lignes, chacune contenant deux entiers $x$ et $y$ ($0 \leq x_i, y_i\leq 10^{9}$) représentant la coordonnée du participant $i$.
    \end{itemize}
\end{Input}

\begin{Output}
Un entier représentant la somme de toutes les distances euclidiennes au carré entre les hubs et les participants. Étant donné que la réponse peut être très grande, vous devez la donner modulo $10^9 + 7$.
\end{Output}
