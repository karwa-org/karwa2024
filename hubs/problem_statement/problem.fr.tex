\problemname{\problemyamlname}

%\illustration{0.3}{image.jpg}{
%    Caption of the illustration (optional).
%    CC BY-SA 4.0 by X on \href{https://example.com/reference-to-image}{Y}
%}

% optionally define variables/limits for this problem
\newcommand{\maxa}{123456789}

% TODO: Remove this comment when you're done writing the problem statement.
Le KARWa a tellement de succès cette année que, l'année prochaine, il sera nécessaire d'organiser $n$ hubs pour
accueillir les $m$ participants. Les hubs sont situés aux coordonnées $(a_i , b_i )$, avec $i = 1, ..., n$ et les participants aux coordonnées $(x_j , y_j )$, avec $j = 1, ..., m$.
Les organisateurs sont des fans du K-Means et afin de les aider à quantifier le placement des hubs, pouvez-vous
déterminer la somme de toutes les distances euclidiennes au carré entre les hubs et les participants ?
\begin{Input}
    L'entrée consiste en :
    \begin{itemize}
        \item Une ligne avec un entier \(n\) (\(1 \leq n \leq 10^{6}\)) représentant le nombre de hubs.
      \item \(n\) lignes contenant chacune deux entiers \(a_{i}\) et \(b_{i}\) séparés par un espace représentant la coordonnée du hub \(i\) avec \(0 \leq a_i, b_i\leq 10^{9}\).
        \item Une ligne avec un entier \(m\) (\(1 \leq m \leq 10^{6}\))  représentant le nombre de participants.
      \item \(m\) lignes contenant chacune deux entiers \(x_{i}\) et \(y_{i}\) séparés par un espace représentant la coordonnée du participant \(i\) avec \(0 \leq x_i, y_i\leq 10^{9}\).
    \end{itemize}
\end{Input}

\begin{Output}
Un entier représentant la somme de toutes les distances euclidiennes au carré entre les hubs et les participants. Étant donné que la réponse peut être très grande, vous devez la donner modulo \(10^9 + 7\).
\end{Output}
