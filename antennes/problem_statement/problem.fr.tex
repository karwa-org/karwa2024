\problemname{\problemyamlname}

\illustration{0.3}{antenna-01_generated.jpg}{
    CC BY-SA 4.0 by X on \href{https://www.vecteezy.com/vector-art/2387825-antenna-icon-flat-style-isolated-on-white-background}{iconvector}
}

La Komission des Antennes Régionnales Wallones (KARWa) a besoin de vous ! En effet, le gouvernement a investi dans énormément de nouveaux bâtiments partout en Wallonie. Bien évidemment, nous sommes en 2024 et ces bâtiments ont besoin d'une connexion internet fiable !

Pour ce faire, la Komission a acheté des antennes à mettre sur les bâtiments. Seulement, il y en a des meilleurs... Ainsi, la Komission fait appel à vous afin de savoir où mettre quelle antenne pour avoir la meilleure couverture possible.

Pour réaliser cela, vous recevez une liste avec le besoin de connectivité de chaque bâtiment. Vous recevez aussi la liste des antennes à placer avec la connectivité qu'elles procurent.

Pour vous simplifier la tâche, votre superviseur vous dit qu'il est suffisant de maximiser la somme des produits entre la connectivité offerte par chaque antenne avec le besoin du bâtiment qui lui est relié. Par exemple, un bâtiment ayant un besoin de $100$ et possédant une antenne offrant une connectivité de $33$ aura un \emph{score} de $3300$. Il vous demande donc quel \emph{score} vous pouvez obtenir pour chaque bâtiment dans le but que la somme de tous les scores soit la plus grande possible. Notez qu'une antenne ne peut être utilisée qu'une seule fois.

\begin{Input}
    L'entrée consiste en :
    \begin{itemize}
      \item une ligne avec un entier $n$ ($1 < N \leq 1000$), le nombre de bâtiments, qui est égal au nombre d'antennes,
      \item $n$ lignes, chacune contenant un entier $b_i$ ($1 \leq b \leq 1000$), le besoin de connectivité nécessaire au bâtiment $i$,
      \item $n$ lignes, chacune contenant un entier $c_i$ ($1 \leq c_i \leq 1000)$, la connectivité offerte par l'antenne $i$.
    \end{itemize}
\end{Input}

\begin{Output}
  $n$ lignes, où la ligne $i$-ème ligne contient un entier $s$, le \emph{score} qu'aura le bâtiment $i$ dans une configuration optimale.
\end{Output}
