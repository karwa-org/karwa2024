\begin{frame}
    \frametitle{\problemtitle}
        \begin{block}
            {Problème} Trouver si le nombre de permutations d'une transposition est pair ou impair.
        \end{block}
        \pause
        \textbf{Solution Naïve}: retirer une à une les permutations de la transposition, c'est compliqué !
\end{frame}

\begin{frame}
    \frametitle{\problemtitle}
    \begin{itemize}
        \item<+-> On peut calculer la signature de la transposition.
        \item<+-> Si la signature vaut $-1$ alors la permutation est impaire, si elle vaut $1$, la permutation est paire.
        \item<+-> La signature d'une transposition $\sigma$ est $\varepsilon(\sigma) = \prod_{1\leq i < j \leq N} \frac{sigma(j)-sigma(i)}{j-i}$.
        \item<+-> Étant donné que l'on s'intéresse au signe de la signature, on peut multiplier par successivement par $-1$  puis $1$ pour éviter les problèmes de précision liés au floats potentiellement grands.
    \end{itemize}
\end{frame}
