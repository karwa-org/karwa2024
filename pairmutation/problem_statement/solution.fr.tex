
\begin{frame}
    \frametitle{\problemtitle}
        % TODO: Remove this comment when you're done writing the solutions.
        \begin{block}
            {Problème} Trouver qui est le meurtrier parmis $n$ personnes.
        \end{block}
        \pause
        \textbf{Solution Naïve}: Tester toutes les paires possibles. $\mathcal O(n^2)$ Trop d'opérations !
\end{frame}

\begin{frame}
    \frametitle{\problemtitle}
    \begin{itemize}
        \item<+-> À la place, on peut voir le problème comme un graphe.
        \item<+-> \textbf{Graphe:} Ensemble de sommet (personnes) reliés par des lignes (A à vu B).
            \\
            \centering
            \begin{figure}%[H]

            \begin{tikzpicture}[>=stealth', shorten >=1pt, auto,
                node distance=2.5cm, scale=1, 
                transform shape, align=center,]

                \begin{scope}[every node/.style={circle,thick,draw, minimum size=1cm}]
                    \node (A) at (0,0) {Alexis};
                    \node (B) at (0,3) {Bob};
                    \node (C) at (2.5,4) {Caro};
                    \node (D) at (5,0) {David};
                    \node (E) at (2.5,-1) {Eliot};
                    \node (F) at (5,3) {Frank} ;
                \end{scope}

                \path[->] (A) edge node {} (C)
                (B) edge node {} (A)
                (D) edge node {} (B)
                (C) edge node {} (D)
                (E) edge node {} (A)
                (E) edge node {} (B)
                (E) edge node {} (C)
                (E) edge node {} (D)
                (E) edge node {} (F);
            
            \end{tikzpicture}
            \end{figure}
        \end{itemize}
\end{frame}

\begin{frame}
    \frametitle{\problemtitle}
    Il y a deux cas possibles:
    \begin{itemize}
        \item<+-> \textbf{Cas 1:} La personne A à vue la personne B.
            \begin{itemize}
                \item Donc la personne B n'est pas le meurtrier.
            \end{itemize}
        \item<+-> \textbf{Cas 2:} La personne A n'a pas vue la personne B.
            \begin{itemize}
                \item Donc la personne B n'est pas le meurtrier.
            \end{itemize}

        \textbf{Solution} Stratégie de \textbf{Diviser pour régner}:
        \begin{itemize}
            \item<+-> On divise notre enssemble de personne en deux.
            \item<+-> On demande à une personne de chaque groupe si elle a vue une personne de l'autre groupe.
            \item<+-> On garde les potentiels meurtriers.
            \item<+-> On répète jusqu'à ce qu'il ne reste que deux personnes.
        \end{itemize}
        Au total on fait $n-1$ opérations. Étant doné qu'a chaque question on élimine une personne. il nous faut donc $n-1$ questions pour trouver le meurtrier.
    \end{itemize}
\end{frame}
