\begin{frame}
    \frametitle{\problemtitle}
        \begin{block}
            {Problème} Trouver deux lignes qui sont identiques en éffectuant des rotations sur les lignes.
        \end{block}
        \pause
        \textbf{Solution Naïve}: Pour chaque ligne, tester si une des autres lignes est identique après une rotation. $\mathcal O(n^2 \cdot m^2)$ Trop lent !\\
        \pause
        \textbf{Solution}: HASH !
        \begin{block} {Hash Polynomial}
            Pour la chaine de caractère \textit{bonjour} on peut le hash de manière suiviante:
            $(b \cdot A^0 + o \cdot A^1 + n \cdot A^2 + j \cdot A^3 + o \cdot A^4 + u \cdot A^5 + r \cdot A^6) \, mod \, B$
        \end{block}
        \pause
        Dès lors on peut créer toutes les rotations d'une ligne en $\mathcal O(n)$.\\
        Il suffit de retirer le dernier caractère et de le mettre au début.
        \pause
        \begin{align*}
           &\rightarrow (b \cdot A^0 + o \cdot A^1 + n \cdot A^2 + j \cdot A^3 + o \cdot A^4 + u \cdot A^5 + r \cdot A^6) \, mod \, B \,\text{(Inital)}\\ 
           &\rightarrow A \cdot (b \cdot A^0 + o \cdot A^1 + n \cdot A^2 + j \cdot A^3 + o \cdot A^4 + u \cdot A^5) \, mod \, B\, \text{(On retire le dernier et multiplie par A)}\\
           &\rightarrow (r \cdot A^0 + b \cdot A^1 + o \cdot A^2 + n \cdot A^3 + j \cdot A^4 + o \cdot A^5 + u \cdot A^6) \, mod \, B\, \text{(On remet au début celui qu'on a retiré)}\\ 
        \end{align*}
        \pause
\end{frame}
\begin{frame}
    \frametitle{\problemtitle}
        \begin{block}
            {Problème} Trouver deux lignes qui sont identiques en éffectuant des rotations sur les lignes.
        \end{block}
    Suffit donc de garder dans un dictionaire les hash et voir si deux hash sont identiques.\\
    \textbf{ATTENTION:} Choisir un bon $B$ pour éviter les collisions. (ici $B$ $\approx$ $2^{50}$ suffit)\\
    \pause
    Complexité: $\mathcal O(n \cdot m)$
\end{frame}
