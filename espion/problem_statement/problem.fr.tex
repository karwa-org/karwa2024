\problemname{\problemyamlname}

\illustration{0.3}{image.jpg}{
    Caption of the illustration (optional).
    CC BY-SA 4.0 by Rizki Aditya Majid on \href{https://www.vecteezy.com/vector-art/35771070-crime-boss-woman-holding-dual-gun}{vecteezy}
}

Vous êtes un agent inflitrée dans une organisation criminelle, En effet un célèbre bandit nomé \textit{John Windows} à capturé les membres du \textit{Louvain-Li-Nux}. Pour ce faire vous vous êtes infliré dans l'endroit où ils sont capturés. Vous avez réussi à les libérer mais vous devez vous échapper avant que les gardes ne vous attrapent.  Vous avez $t$ secondes pour vous échapper avant d'être attrapé.

Cependant sur le chemin de la sortie se trouve $n$ gardes. Vous avez le choix entre les neutraliser ou les éviter. Vous possedez deux armes, un pistolet et un couteau. Le pistolet prends $p_i$ temps pour neutraliser le guarde $i$, le couteau prends $c_i$ temps pour neutraliser le guarde $i$ et l'equiver prends $e_i$ temps pour neutrealiser le guarde $i$. Attention vous ne pouvez pas utilise deux fois la même arme à la suite !.

Dès lors vous voulez savoir le nombre ($x$) maximum de garde que vous pouvez neutraliser avant de vous échapper ainsi que le temps minimum qu'il vous faut pour échaper les $x$ guardes.
\begin{Input}
    L'entrée consiste en:
    \begin{itemize}
        \item une ligne avec deux entiers $n$ ($1\leq n \leq 5 \cdot 10^5$), le nombre de garde et $t$ ($1\leq t \leq 10^9$), le temps que vous avez pour vous échapper.
        \item $n$ lignes avec trois entiers  $p_i$ ($1\leq p_i \leq 10^9$), $c_i$ ($1\leq c_i \leq 10^9$) et $e_i$ ($1\leq e_i \leq 10^9$), le temps que prends chaque arme pour neutraliser le garde $i$.
    \end{itemize}
\end{Input}

\begin{Output}
    Sortez un entier $x$, le nombre de garde que vous pouvez neutraliser avant de vous échapper et un entier $m$ le temps minimum qu'il vous faut pour échaper les $x$ guardes.
\end{Output}
