\problemname{\problemyamlname}

\illustration{0.3}{image.jpg}{
    CC BY-SA 4.0 by Rizki Aditya Majid on \href{https://www.vecteezy.com/vector-art/35771070-crime-boss-woman-holding-dual-gun}{vecteezy}
}

Vous êtes un agent infiltré dans une organisation criminelle. En effet, un célèbre bandit nommé \textit{John Windows} a capturé les membres du \textit{Louvain-Li-Nux}. Pour ce faire, vous vous êtes infiltré dans l'endroit où ils sont capturés. Vous avez réussi à les libérer, mais vous devez vous échapper avant que les gardes ne vous attrapent.  Vous avez $t$ secondes pour vous échapper avant d'être attrapé.

Cependant, sur le chemin de la sortie se trouvent $n$ gardes. Vous avez le choix entre les neutraliser ou les éviter. Vous possédez deux armes, un pistolet et un couteau. Le pistolet prend $p_i$ secondes pour neutraliser le garde $i$, le couteau prend $c_i$ secondes pour neutraliser le garde $i$ et esquiver le garde $i$ prend $e_i$ secondes. Attention, vous ne pouvez pas utiliser deux fois la même arme à la suite !

Dès lors, vous voulez savoir le nombre ($x$) maximum de garde que vous pouvez neutraliser avant de vous échapper ainsi que le temps minimum qu'il vous faut pour échapper les $x$ gardes.
\begin{Input}
    L'entrée consiste en:
    \begin{itemize}
        \item une ligne avec deux entiers $n$ et $t$ ($1\leq n \leq 5 \cdot 10^5$, $1\leq t \leq 10^9$), le nombre de gardes et le temps que vous avez pour vous échapper.
        \item $n$ lignes, la $i$-ème ligne contenant trois entiers  $p_i$, $c_i$ et $e_i$ ($1\leq p_i, c_i, e_i \leq 10^9$), le temps que prend chaque arme pour neutraliser le garde $i$.
    \end{itemize}
\end{Input}

\begin{Output}
    Sortez un entier $x$, le nombre de gardes que vous pouvez neutraliser avant de vous échapper et un entier $m$ le temps minimum qu'il vous faut pour échapper les $x$ gardes.
\end{Output}
