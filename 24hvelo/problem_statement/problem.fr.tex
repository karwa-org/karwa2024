\problemname{\problemyamlname}

\newcommand{\maxk}{3 \cdot 10^5}
\newcommand{\maxn}{64}

\illustration{0.3}{image.jpg}{
    Légende de l'illustration
}
Après les 24h vélo, Alexandre fait ses comptes. Malheureusement, il ne se souvient plus de ce qu'il a acheté car il a perdu ses tickets de caisse. Heureusement, il se rappelle des prix des objets et du montant total qu'il a payé. Pouvez-vous retrouver quels objets il a achetés uniquement avec le montant total ? S'il n'existe aucune solution, retournez -1. Il est garanti que les objets sont uniques, et il n'est pas possible qu'Alexandre ait acheté plusieurs fois le même objet. \\
S'il existe plusieurs solutions, affichez-en une.

\begin{Input}
    L'entrée consiste en :
    \begin{itemize}
        \item Une ligne avec deux entiers $n, k$ ($0 \leq n \leq \maxn$), $(0 \leq k \leq \maxk)$ représentant le nombre d'objets et le montant total.
        \item Une ligne avec $n$ objets. Chaque objet a un nom d'au plus $10^4$ caractères sans espaces et un prix $p$ qui est un entier. $(0 \leq p \leq \maxk)$
    \end{itemize}
    Il est garantis que la somme des prix des objets est d'au plus $\maxk$.
\end{Input}

\begin{Output}
    Sortez le nombre d'objets qu'Alexandre a achetés suivis des noms des objets. Sortez $-1$ s'il n'y a aucun objet correspondant au montant total.
\end{Output}
