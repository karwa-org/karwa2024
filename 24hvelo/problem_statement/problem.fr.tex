\problemname{\problemyamlname}

\newcommand{\maxk}{3 \cdot 10^5}
\newcommand{\maxn}{64}

\illustration{0.3}{image.jpg}{

}
Après les 24h vélo, Alexandre fait ses comptes. Malheureusement, il ne se souvient plus de ce qu'il a acheté car il a perdu ses tickets de caisse. Heureusement, il se rappelle des prix des objets et du montant total qu'il a payé. Pouvez-vous retrouver quels objets il a achetés uniquement avec le montant total ? S'il n'existe aucune solution, retournez \texttt{-1}. Il est garanti que les objets sont uniques, et il n'est pas possible qu'Alexandre ait acheté plusieurs fois le même objet. \\
S'il existe plusieurs solutions, affichez-en une.

\begin{Input}
    L'entrée consiste en :
    \begin{itemize}
        \item une ligne avec deux entiers $n$ et $k$ ($0 \leq n \leq \maxn$, $0 \leq k \leq \maxk)$ représentant le nombre d'objets et le montant total payé,
        \item $n$ lignes, chacune contenant $s$ et $p$, où $s$ est une chaîne de caractères de longueur au plus $10^4$ donnant le nom d'un objet et $p$ est son prix $(0 \leq p \leq \maxk)$.
    \end{itemize}
    Il est garanti que la somme des prix des objets est d'au plus $\maxk$.
\end{Input}

\begin{Output}
    Sortez le nombre $m$ d'objets qu'Alexandre a achetés, suivi de $m$ lignes, chacune avec le nom d'un objet. Sortez \texttt{-1} s'il n'y a aucun objet correspondant au montant total.
\end{Output}
