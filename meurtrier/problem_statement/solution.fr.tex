\definecolor{ggreen}{HTML}{30b020}

\begin{frame}
    \frametitle{\problemtitle}
    \begin{block}{Problème}
        Trouver qui est le meurtrier parmi $n$ personnes.
    \end{block}
    \pause
    \textbf{Solution Naïve} : Tester toutes les paires possibles. $\mathcal O(n^2)$ Trop d'opérations !
    \pause
    \begin{itemize}
        \item<+-> À la place, on peut voir le problème comme un graphe.
        \item<+-> \textbf{Graphe :} Ensemble de sommets (personnes) reliés par des lignes (A a vu B).
            \\ \medskip
            \centering
            \begin{figure}

            \begin{tikzpicture}[>=stealth', shorten >=1pt, auto,
                node distance=2.5cm, scale=1,
                transform shape, align=center,scale=0.75]

                \begin{scope}[every node/.style={circle,thick,draw, minimum size=1cm}]
                    \node (A) at (0,0) {Alexis};
                    \node (B) at (0,3) {Bob};
                    \node (C) at (2.5,4) {Caro};
                    \node (D) at (5,0) {David};
                    \node (E) at (2.5,-1) {Eliot};
                    \node (F) at (5,3) {Frank} ;
                \end{scope}

                \path[->] (A) edge node {} (C)
                (B) edge node {} (A)
                (D) edge node {} (B)
                (C) edge node {} (D)
                (E) edge node {} (A)
                (E) edge node {} (B)
                (E) edge node {} (C)
                (E) edge node {} (D)
                (E) edge node {} (F);

            \end{tikzpicture}
            \end{figure}
    \end{itemize}
\end{frame}

\begin{frame}
    \frametitle{\problemtitle}
    Il y a deux cas possibles :
    \begin{itemize}
        \item<+-> \textbf{Cas 1 :} La personne A a vu la personne B.
            \begin{itemize}
                \item<+-> Donc la personne B n'est pas le meurtrier.
            \end{itemize}
        \item<+-> \textbf{Cas 2 :} La personne A n'a pas vu la personne B.
            \begin{itemize}
                \item<+-> Donc la personne A n'est pas le meurtrier.
            \end{itemize}
    \end{itemize}
    \pause
    \textbf{Solution} : On Applique la règle ci-dessus tout en éliminant les suspects petit à petit ce qui nous donne $n-1$ opérations. (Attention au cas $n = 1$ !)
\end{frame}

\begin{frame}\frametitle{\problemtitle}
    Est-ce que le nœud bleu à vu le nœud vert ?
    \centering
    \begin{figure}%[H]
        \begin{tikzpicture}[>=stealth', shorten >=1pt, auto,
            node distance=2.5cm, scale=1,
            transform shape, align=center,every node/.style={circle,thick,draw, minimum size=1cm}]

            \node (A)[ggreen] at (0,0) {Alexis};
            \node (B)[blue] at (0,3) {Bob};
            \node (C) at (2.5,4) {Caro};
            \node (D) at (5,0) {David};
            \node (E) at (2.5,-1) {Eliot};
            \node (F) at (5,3) {Frank};

            \path[->] (A) edge (C)
            (B) edge (A)
            (D) edge (B)
            (C) edge (D)
            (E) edge (A)
            (E) edge (B)
            (E) edge (C)
            (E) edge (D)
            (E) edge (F);

        \end{tikzpicture}
    \end{figure}
\end{frame}


\begin{frame}\frametitle{\problemtitle}
    Est-ce que le nœud bleu à vu le nœud vert ?
    \centering
    \begin{figure}%[H]
        \begin{tikzpicture}[>=stealth', shorten >=1pt, auto,
            node distance=2.5cm, scale=1,
            transform shape, align=center,every node/.style={circle,thick,draw, minimum size=1cm}]
            \node (A)[red] at (0,0) {\Huge X};
            \node (B)[blue] at (0,3) {Bob};
            \node (C)[ggreen] at (2.5,4) {Caro};
            \node (D) at (5,0) {David};
            \node (E) at (2.5,-1) {Eliot};
            \node (F) at (5,3) {Frank};

            \path[->] (A) edge (C)
            (B) edge (A)
            (D) edge (B)
            (C) edge (D)
            (E) edge (A)
            (E) edge (B)
            (E) edge (C)
            (E) edge (D)
            (E) edge (F);

        \end{tikzpicture}
    \end{figure}
\end{frame}


\begin{frame}\frametitle{\problemtitle}
    Est-ce que le nœud bleu à vu le nœud vert ?
    \centering
    \begin{figure}%[H]
        \begin{tikzpicture}[>=stealth', shorten >=1pt, auto,
            node distance=2.5cm, scale=1,
            transform shape, align=center,every node/.style={circle,thick,draw, minimum size=1cm}]
            \node (A)[red] at (0,0) {\Huge X};
            \node (B)[red] at (0,3) {\Huge X};
            \node (C)[blue] at (2.5,4) {Caro};
            \node (D)[ggreen] at (5,0) {David};
            \node (E) at (2.5,-1) {Eliot};
            \node (F) at (5,3) {Frank};

            \path[->] (A) edge (C)
            (B) edge (A)
            (D) edge (B)
            (C) edge (D)
            (E) edge (A)
            (E) edge (B)
            (E) edge (C)
            (E) edge (D)
            (E) edge (F);

        \end{tikzpicture}
    \end{figure}
\end{frame}

\begin{frame}\frametitle{\problemtitle}
    Est-ce que le nœud bleu à vu le nœud vert ?
    \centering
    \begin{figure}%[H]
        \begin{tikzpicture}[>=stealth', shorten >=1pt, auto,
            node distance=2.5cm, scale=1,
            transform shape, align=center,every node/.style={circle,thick,draw, minimum size=1cm}]
            \node (A)[red] at (0,0) {\Huge X};
            \node (B)[red] at (0,3) {\Huge X};
            \node (C)[blue] at (2.5,4) {Caro};
            \node (D)[red] at (5,0) {\Huge X};
            \node (E) at (2.5,-1) {Eliot};
            \node (F)[ggreen] at (5,3) {Frank};

            \path[->] (A) edge (C)
            (B) edge (A)
            (D) edge (B)
            (C) edge (D)
            (E) edge (A)
            (E) edge (B)
            (E) edge (C)
            (E) edge (D)
            (E) edge (F);

        \end{tikzpicture}
    \end{figure}
\end{frame}

\begin{frame}\frametitle{\problemtitle}
    Est-ce que le nœud bleu à vu le nœud vert ?
    \centering
    \begin{figure}%[H]
        \begin{tikzpicture}[>=stealth', shorten >=1pt, auto,
            node distance=2.5cm, scale=1,
            transform shape, align=center,every node/.style={circle,thick,draw, minimum size=1cm}]
            \node (A)[red] at (0,0) {\Huge X};
            \node (B)[red] at (0,3) {\Huge X};
            \node (C)[red] at (2.5,4) {\Huge X};
            \node (D)[red] at (5,0) {\Huge X};
            \node (E)[ggreen] at (2.5,-1) {Eliot};
            \node (F)[blue] at (5,3) {Frank};

            \path[->] (A) edge (C)
            (B) edge (A)
            (D) edge (B)
            (C) edge (D)
            (E) edge (A)
            (E) edge (B)
            (E) edge (C)
            (E) edge (D)
            (E) edge (F);

        \end{tikzpicture}
    \end{figure}
\end{frame}

\begin{frame}\frametitle{\problemtitle}
    Est-ce que le nœud bleu à vu le nœud vert ?
    \begin{center}
    \begin{figure}%[H]
        \begin{tikzpicture}[>=stealth', shorten >=1pt, auto,
            node distance=2.5cm, scale=1,
            transform shape, align=center,every node/.style={circle,thick,draw, minimum size=1cm}]
            \node (A)[red] at (0,0) {\Huge X};
            \node (B)[red] at (0,3) {\Huge X};
            \node (C)[red] at (2.5,4) {\Huge X};
            \node (D)[red] at (5,0) {\Huge X};
            \node (E)[ggreen] at (2.5,-1) {Eliot};
            \node (F)[red] at (5,3) {\Huge X};

            \path[->] (A) edge (C)
            (B) edge (A)
            (D) edge (B)
            (C) edge (D)
            (E) edge (A)
            (E) edge (B)
            (E) edge (C)
            (E) edge (D)
            (E) edge (F);

        \end{tikzpicture}
    \end{figure}
    \end{center}

    \pause\solvestats
\end{frame}