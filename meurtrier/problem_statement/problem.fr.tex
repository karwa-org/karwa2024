\problemname{\problemyamlname}

\illustration{0.3}{murder.jpg}{
    Caption of the illustration (optional).
    CC BY-SA 4.0 by irkhamgram on \href{https://www.vecteezy.com/vector-art/12335116-illustration-of-serial-killer-with-dagger-psychopath-spooky-the-background-is-filled-with-blood-suitable-for-halloween-themes-horror-movies-scary-murder-crime-flat-vector}{vecteezy}
}

\newcommand{\maxn}{10e4}  % max number of people

On vous a appelé pour enquêter sur un crime qui a eu lieu dans un grand manoir. Quelqu'un a tué Mr Bob ! Fin détective comme vous êtes, vous avez déjà émis une hypothèse,
Le meurtrier ne s'est pas fait voir par les autres membres de la maison avant le meurte. Étant donné que pour planifier un meurtre il faut être discret.

Vous savez aussi que le meurtrier à du voir tout le monde pour planifier son meurtre. (Savoir si albert est dans la cuisine ou bien si michelle dans sa chambre).

Dès lors vous pouvez intéroger les $n$ personnes qui étaient dans la maison au moment du meurtre. 


\section*{Interaction}
Vous allez intéragir 

Votre programme doit retourner le nom du meurtrier. Pour ce faire, vous allez intéragir avec un programme.

En entrée vous aurez une ligne qui contient un entier $n$ le nombre de personne.
Ensuite $n$ lignes avec le nom des personnes présent dans le manoire. Il est garantis que le nom des personnent sont distincts et contiennent uniquement des lettres alphanumériques.
Il est aussi garantis que la taille des noms et d'au maximum $1000$.

Vous pouvez retourner une ligne avec l'un de ces mots:

\begin{itemize}
    \item ``\texttt{? a b}'' demandez à $a$ si il a vu $b$.
    \item ``\texttt{! a}'' la perssonne $a$ est le meurtrier.
\end{itemize}

L'interacteur vous répondera avec une ligne avec un de ces mots:
\begin{itemize}
    \item ``\texttt{OUI}'' si la personne $a$ à vu la personne $b$.
    \item ``\texttt{NON}'' si la personne $a$ n'a pas vu la personne $b$.
\end{itemize}

Une fois un personne acussée, votre programme devra se terminer.
Vous pouvez faire au maximum $n - 1$ questions. Si vous faites plus de questions, votre programme sera terminé et sera considéré comme faux. 
%TODO : PROOF N - 1