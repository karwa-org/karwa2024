\begin{frame}
    \frametitle{\problemtitle}
    \begin{itemize}
        \item<+-> \textbf{Problème:} Accessibilité (quantitative) dans un jeu sur graphe, à somme nulle et à deux joueurs !
        \item<+-> On a un graphe dirigé $(V,E)$, sommets $V_1$ représentant les gares dont le passager n'a aucun contrôle et $V_0$ les autres. On veut calculer en combien de coût on peut atteindre la destination si on joue optimalement et que l'adversaire joue également optimalement pour nous retarder au maximum.
        \item<+-> Résolvable par un algorithme d'attracteur (peut aussi être vu comme un MIN-MAX): $\mathcal{O}(|V| + |E|).
    \end{itemize}
    \solvestats
\end{frame}
