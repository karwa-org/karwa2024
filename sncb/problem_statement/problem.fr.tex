\problemname{\problemyamlname}

\begin{wrapfigure}{r}{5.5cm}
    \centering
    \includegraphics[width=5.5cm]{mirror.jpg}
\end{wrapfigure}

Vous souhaitez vous rendre à votre compétition d'algorithmique préférée qui commence ce soir ! Problème, vous vous y rendez en train, et la SNCB a tout juste déployé son plan \emph{Kilomètres À Rallonge, Wagons en Avarie} (KARWA), visant à maximiser le retard ferroviaire des belges. Ce plan est très simple: certaines gares sont complètement imprévisibles et la SNCB peut vous emmener à un lieu non-désiré, n'offrant aucun contrôle aux voyageurs de la destination. Heureusement pour vous, ce n'est pas encore le cas de toutes les gares.

La traversée de la Belgique s'avère donc compliqué. Chaque gare possède un certain nombre d'autres gares adjacentes (au moins une). Certaines gares appliquent déjà le plan KARWA, auquel cas votre train peut se retrouver à n'importe quelle gare adjacente, au bon vouloir de la SNCB, tandis que lorsque la gare n'applique pas le plan, c'est vous qui avez le choix de la destination à la gare suivante. Attention, certaines voies sont unidirectionnelles ! C'est-à-dire qu'il est possible que vous puissiez aller de A à B, mais pas de B à A.

Votre but est de minimiser le nombre de correspondances de votre point de départ à votre destination. Vous démarrez toujours de la gare $0$ et devez atteindre la gare $1$. Si vous avez une seule gare intermédiaire, par exemple $0 \rightarrow 2 \rightarrow 1$, alors vous aurez une seule correspondance.

\begin{Input}
	L'entrée consiste en :
	\begin{itemize}
		\item une ligne contenant deux entiers $n$ ($0 < n \le 10^6$) et $m$ ($0 < m \le 10^6$), respectivement le nombre de gares belges sur le réseau et le nombre de liaisons entre deux gares,
		\item une ligne de $n$ Booléens, la $i$-ème valeur vaut $1$ si et seulement si la gare numéro $i$ applique le plan KARWA,
		\item $m$ lignes où chaque ligne contient deux entiers $i, j$ ($0 \leq i,j < n$), indiquant que la gare numéro $i$ a une voie lui permettant d'aller à la gare numéro $j$.
	\end{itemize}
\end{Input}

\begin{Output}
	Le plus petit nombre de correspondances afin de rejoindre la gare $1$ en partant de la gare $0$, ou ``\texttt{IMPOSSIBLE}'' si la SNCB peut ne jamais vous faire atteindre la gare $1$.
\end{Output}
