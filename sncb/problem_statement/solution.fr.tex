\usetikzlibrary{overlay-beamer-styles}

\begin{frame}
    \frametitle{\problemtitle}
     \begin{block}{Problème}
        Min-Max dans un graphe avec des poids de $1$ sur chaque arête (Théorie des jeux : résoudre un jeu sur graphe d'accessibilité (quantitative) à somme nulle et à information parfaite $\rightsquigarrow$ Attracteur).
    \end{block}
    Plusieurs approches possibles ! En voici une.
    \pause
    \begin{itemize}
       \item<+-> On a un graphe dirigé $(V,E)$ avec $V_0$ les sommets qu'on contrôle et $V_1$ les sommets qu'on ne contrôle pas.
       \item<+-> Donner une valeur $\lambda_0(v)$ à chaque sommet $v \in V$, initialement $\lambda_0(v) = +\infty$ si $v \neq 1$ et $\lambda_0(v) = 0$ (si on démarre à l'arrivée, on a 0 correspondance).
       \item<+-> Définir pour $k \in \mathbb{N}$ :
          \begin{align*}
          \lambda_{k+1}(v)&=\begin{cases}
              \underset{v'\in\mathsf{succ}(v)}{\min} \lambda_{k}(v')+1 &\text{si } v\in V_0\\
              \underset{v'\in\mathsf{succ}(v)}{\max} \lambda_{k}(v')+1 &\text{sinon}.
          \end{cases}
          \end{align*}
          et calculer jusqu'à ce que $\lambda_{k+1} = \lambda_{k}$, puis afficher $\lambda_{k}(0)$ (le mieux qu'on puisse avoir depuis le sommet $0$ quand l'adversaire joue la pire situation pour nous).
     \end{itemize}
\end{frame}

\begin{frame}
    \frametitle{\problemtitle}
     \begin{align*}
          \lambda_{k+1}(v)&=\begin{cases}
              \underset{v'\in\mathsf{succ}(v)}{\min} \lambda_{k}(v')+1 &\text{si } v\in V_0\\
              \underset{v'\in\mathsf{succ}(v)}{\max} \lambda_{k}(v')+1 &\text{sinon}.
          \end{cases}
     \end{align*}
     \pause

     \begin{figure}
     \centering
     \begin{tikzpicture}[->,>=stealth,shorten >=1pt,on grid,node distance=2cm,semithick,double distance=1.5pt,scale=0.6,every node/.style={draw,scale=0.6,minimum size=0mm}]
      \node[circle,inner sep=5pt,fill=gray!20,fill on=<6->,label=above:{$+\infty$}]                                    (v0)              {$v_0$};
      \node[rectangle,inner sep=8pt,fill=gray!20,fill on=<6->,label=above:{$+\infty$}]                               (v1)[right of=v0] {$v_4$};
      \node[circle,inner sep=5pt,fill=gray!20,fill on=<3->,label=above right:{\alt<3->{$1$}{$+\infty$}}]          (v2)[right of=v1] {$v_2$};
      \node[rectangle,inner sep=8pt,fill=gray!20,fill on=<6->,label=left:{$+\infty$}]                               (v3)[below of=v0] {$v_3$};
      \node[circle,inner sep=5pt,fill=gray!20,label=above right:{$0$}](v4)[below of=v2] {$v_1$};
      \node[circle,inner sep=5pt,fill=gray!20,fill on=<5->,label=left:{\alt<5->{$3$}{$+\infty$}}]          (v5)[below of=v3] {$v_5$};
      \node[rectangle,inner sep=8pt,fill=gray!20,fill on=<4->,label=above:{\alt<4->{$2$}{$+\infty$}}]     (v6)[right of=v5] {$v_6$};
      \node[circle,inner sep=5pt,fill=gray!20,fill on=<3->,label=above right:{\alt<3->{$1$}{$+\infty$}}]          (v7)[right of=v6] {$v_7$};

      \path (v0) edge                (v1)
                 edge[bend right=22] (v3)
            (v1) edge[background default aspect={semithick},background aspect={ultra thick},aspect on=<6->] (v3)
                 edge                (v2)
            (v2) edge[background default aspect={semithick},background aspect={ultra thick},aspect on=<3->] (v4)
            (v3) edge[bend right=22,background default aspect={semithick},background aspect={ultra thick},aspect on=<6->] (v0)
                 edge[bend right=22] (v5)
            (v4) edge                (v1)
                 edge[loop right]    (v4)
            (v5) edge[background default aspect={semithick},background aspect={ultra thick},aspect on=<5->] (v6)
                 edge[bend right=22] (v3)
            (v6) edge[bend right=22,background default aspect={semithick},background aspect={ultra thick},aspect on=<4->] (v7)
                 edge                (v4)
            (v7) edge[bend right=22] (v6)
                 edge[background default aspect={semithick},background aspect={ultra thick},aspect on=<3->] (v4);
     \end{tikzpicture}
     \end{figure}

     \pause\pause\pause\pause\pause
     $\rightsquigarrow \mathcal{O}(|V| + |E|)$.
     \pause\solvestats
\end{frame}

