\begin{frame}
    \frametitle{\problemtitle}
    \begin{itemize}
        \item<+-> \textbf{Problème:} Accessibilité (quantitative) dans un jeu sur graphe, à somme nulle et à deux joueurs !
        \item<+-> On a un graphe dirigé $(V,E)$, sommets $V_1$ représentant les gares dont le passager n'a aucun contrôle et $V_0$ les autres. On veut calculer en combien de coût on peut atteindre la destination si on joue optimalement et que l'adversaire joue également optimalement pour nous retarder au maximum.

        \begin{figure}
        \centering
        \begin{tikzpicture}[->,>=stealth,shorten >=1pt,on grid,node distance=2cm,semithick,double distance=1.5pt,scale=0.6,every node/.style={scale=0.6}]
          \node[circle,draw,minimum size=0mm,inner sep=5pt,label=above:{$+\infty$}]                       (v0)              {$v_0$};
          \node[rectangle,draw,minimum size=0mm,inner sep=8pt,label=above:{$+\infty$}]                  (v1)[right of=v0] {$v_1$};
          \node[circle,draw,minimum size=0mm,inner sep=5pt,fill=gray!20,label=above:{$1$}]                (v2)[right of=v1] {$v_2$};
          \node[rectangle,draw,minimum size=0mm,inner sep=8pt,label=right:{$+\infty$}]                  (v3)[below of=v0] {$v_3$};
          \node[circle,draw,minimum size=0mm,inner sep=5pt,fill=gray!20,label=above right:{$0$}](v4)[below of=v2] {$T$};
          \node[circle,draw,minimum size=0mm,inner sep=5pt,fill=gray!20,label=above right:{$3$}]          (v5)[below of=v3] {$v_5$};
          \node[rectangle,draw,minimum size=0mm,inner sep=8pt,fill=gray!20,label=above:{$2$}]           (v6)[right of=v5] {$v_6$};
          \node[circle,draw,minimum size=0mm,inner sep=5pt,fill=gray!20,label=right:{$1$}]                (v7)[right of=v6] {$v_7$};

          \path (v0) edge  (v1)
                     edge[bend right=22](v3)
                (v1) edge               (v3)
                     edge               (v2)
                (v2) edge  (v4)
                (v3) edge[bend right=22](v0)
                     edge[bend right=22](v5)
                (v4) edge  (v1)
                     edge[loop right]   (v4)
                (v5) edge  (v6)
                     edge[bend right=22](v3)
                (v6) edge[bend right=22](v7)
                     edge               (v4)
                (v7) edge[bend right=22](v6)
                     edge  (v4);

        \end{tikzpicture}
        \end{figure}

        \item<+-> Résolvable par un algorithme d'attracteur (peut aussi être vu comme un MIN-MAX): $\mathcal{O}(|V| + |E|)$.
    \end{itemize}
    %\solvestats
\end{frame}
