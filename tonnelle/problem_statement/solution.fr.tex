\begin{frame}
    \frametitle{\problemtitle}
    \begin{block}{Problème}
        On veut calculer l'aire du plus petit rectangle contenant un ensemble de $n$ points.
    \end{block}
    \pause
    Pour chaque direction, le plus petit rectangle aligné avec cette direction s'obtient en trouvant les points extrêmes selon cette direction et son orthogonale. \newline \newline
    \textbf{Solution naïve} : Prendre le plus petit rectangle aligné verticalement. \\
    \pause
    Cette approche ne marche malheureusement pas en général, par exemple pour un carré penché.
\end{frame}

\begin{frame}
    \frametitle{\problemtitle}
    \textbf{Solution} : Prendre le plus petit rectangle aligné avec un segment de l'enveloppe convexe des points. \newline \newline
    \pause
    Il est inutile de regarder toutes les directions possibles, puisque le plus petit rectangle aura un côté confondu avec un des segments de l'enveloppe convexe des points. \newline \newline
    \pause
    L'enveloppe convexe peut se calculer en $\mathcal{O}(n \log(n))$, mais une implémentation naïve en $\mathcal{O}(n^2)$ est ici suffisante car tester la direction donnée par chaque segment de l'enveloppe convexe fournit un code en $\mathcal{O}(n^2)$.
\end{frame}